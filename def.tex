\documentclass {article}
\usepackage{fullpage}
\usepackage{amsmath}
\usepackage{ulem}
\usepackage[all]{xy}

\providecommand{\abs}[1]{\lvert#1\rvert}
\providecommand{\norm}[1]{\lVert#1\rVert}

\begin {document}
\normalem  % Overrides the ulem package to preserve italics based emphasis.

\section{DeBoor definition}

\begin{equation}
\lbrace \Omega \rbrace_{jk} = \int B_j^{\prime\prime}(t) B_k^{\prime\prime}(t) dt
\end{equation}

\begin{equation}
[ \tau_i,...,\tau_{i+r}]g = \frac{\tau_{i+1},...,\tau_{i+r}]g - \tau_i,...,\tau_{i+r-1}]g}{\tau_{i+r}-\tau_{i}}
\end{equation}

\begin{equation}
B_{j,k,t}(x) := (t_{j+k}-t_j)[t_j,...,t_{j+k}](\cdot - x)_+^{k-1}
\end{equation}

\section{Hastie definition}
An order-M spline with knots \({\xi}_j\), \(j=1,...,K\) is a piece-wise polynomial of order \(M\),
and has continuous derivatives up to order \(M-2\). A cubic spline has order \(M=4\).
\begin{equation}
\begin{array}{l}
\tau_1 \le \tau_2 \le \dotsi \le \tau_M \le \xi_0 \\
\tau_{j+M} = \xi_j, j=1,\dotsi,K \\
\xi_{K+1} \le \tau_{K+M+1} \le \tau_{K+M+2} \le \dotsi \le \tau_{K+2M}
\end{array}
\end{equation}

\begin{equation}
	B_{i,1}(x)
	= \left\{
  \begin{array}{l l}
  	1 & \text{if } \tau_i \le x < \tau_{i+1} \\
    0 & \text{otherwise} \\
  \end{array} \right.
\end{equation}
for \(i=1,...,K+2M-1\). (Haar functions.)

\begin{equation}
	B_{i,m}(x) = \frac{x-\tau_i}{\tau_{i+m-1}-\tau_i}B_{i,m-1}(x) + \frac{\tau_{i+m}-x}{\tau_{i+m}-\tau_{i+1}}B_{i+1,m-1}(x)
\end{equation}
for \(i=1,...,K+2M-m\).

\section{Hastie smoothing}
\begin{equation}
	RSS(f,\lambda) = \sum\limits_{i=1}^N \lbrace y_i-f(x_i)\rbrace^2 + \lambda \int\lbrace f^{\prime\prime}(t)\rbrace^2 dt
\end{equation}

\begin{equation}
	f(x) = \sum\limits_{j=1}^N N_j(x)\theta_j
\end{equation}

\begin{equation}
	RSS(\theta,\lambda) = (\mathbf{y} - \mathbf{N}\theta)^T(\mathbf{y} - \mathbf{N}\theta) + \lambda\theta^T\mathbf{\Omega}_N\theta
%	RSS(\theta_p,\lambda) = (\mathbf{y} - \mathbf{N_p}\theta_p)^T(\mathbf{y} - \mathbf{N_p}\theta_p) + \lambda\theta_p^T\mathbf{\Omega}_{N_p}\theta_p
\end{equation}
...where \(\lbrace\mathbf{N}\rbrace_{ij} = N_j(x_i)\) and \(\lbrace\mathbf{\Omega}_N\rbrace_{jk} = \int N_j^{\prime\prime}(t)N_k^{\prime\prime}(t)dt\).

\begin{equation}
\hat{\theta} = (\mathbf{N}^T\mathbf{N} + \lambda\mathbf{\Omega}_N)^{-1}\mathbf{N}^T\mathbf{y}
\end{equation}
\begin{equation}
\hat{f}(x) = \sum\limits_{j=1}^N N_j(x)\hat{\theta}_j
%\hat{f}_p(x) = \sum\limits_{j=1}^N N_{pj}(x)\hat{\theta}_{pj}
\end{equation}

\begin{equation}
\hat{f}^\prime(x) = \sum\limits_{j=1}^N N_j^\prime(x)\hat{\theta}_j
\end{equation}

From here on we need to deal with multiple spline fits to different probes' time series.
\begin{equation}
	f_p(x) = \sum\limits_{j=1}^N N_j(x)\theta_{pj}
\end{equation}
Notice that the basis need not change. It is determined solely by the knots (time points).
\begin{equation}
\begin{split}
D(p, q ) &= \int \abs{\hat{f}_p^\prime(x)-\hat{f}_q^\prime(x)} dx \\
&= \int \abs{\sum\limits_{j=1}^N N_j^\prime(x)\theta_{pj}-\sum\limits_{j=1}^N N_j^\prime(x)\theta_{qj}} dx \\
&= \sum\limits_{i=0}^4 \int\limits_{\xi_i}^{\xi_{i+1}} \abs{\sum\limits_{j=i}^{i+3} N_j^\prime(x)\theta_{pj}-\sum\limits_{j=i}^{i+3} N_j^\prime(x)\theta_{qj}} dx \\
&= \sum\limits_{i=0}^4 \int\limits_{\xi_i}^{\xi_{i+1}} \abs{\sum\limits_{j=i}^{i+3} (\theta_{pj}-\theta_{qj})N_j^\prime(x)} dx \\
&= \sum\limits_{i=0}^4 \left\{ 
	\int\limits_{\xi_i}^{\alpha_i} \sum\limits_{j=i}^{i+3} (\theta_{pj}-\theta_{qj})N_j^\prime(x) dx 
	+ \int\limits_{\alpha_i}^{\beta_i} \sum\limits_{j=i}^{i+3} (\theta_{pj}-\theta_{qj})N_j^\prime(x) dx 
	+ \int\limits_{\beta_i}^{\xi_{i+1}} \sum\limits_{j=i}^{i+3} (\theta_{pj}-\theta_{qj})N_j^\prime(x) dx 
	\right\}
\end{split}
\end{equation}
Because the \(N_j^\prime(x)\) are quadratics, in each interval there can be no more than 2 curve crossings,
so it is possible to break up the full integral into a small number of intervals (at most \(3k\)) on each
of which \(D\) is exactly solvable.
\newpage
\section{Implementation guides}
\begin{center}
\begin{tabular}{r|c c c c c}
 & \multicolumn{5}{c}{Interval} \\
       & 0 & 1 & 2 & 3 & 4 \\
Spline & [0,1) & [1,2) & [2,4) & [4,6) & [6,12] \\
\hline
0 & x & . & . & . & . \\
1 & x & x & . & . & . \\
2 & x & x & x & . & . \\
3 & x & x & x & x & . \\
4 & . & x & x & x & x \\
5 & . & . & x & x & x \\
6 & . & . & . & x & x \\
7 & . & . & . & . & x \\
\end{tabular}
\end{center}

Following shows how the various polynomial errors cover intervals.
Notice that the left and right 3 intervals are not real integers.
\\
\\
\\
\xymatrix @C=2mm@R=3mm{ 
fxn     & & & & 0 \ar@{-}[d] & & 1\ar@{-}[d] & & 2\ar@{-}[d] & & 3\ar@{-}[d] & & 4\ar@{-}[d] & & 5\ar@{-}[d] & & 6\ar@{-}[d] & & 7\ar@{-}[d]   \\
4 & & & & \ar[dl] \ar[dr] & & & & & & & & & & & & & & \ar[dl] \ar[dr] \\
3 & & & \ar[dl] \ar[dr] & & \ar[dl] \ar[dr] & & & & & & & & & & & & \ar[dl] \ar[dr] & & \ar[dl] \ar[dr] \\
2 & & \ar[dl] \ar[dr] & & \ar[dl] \ar[dr] & & \ar[dl] \ar[dr] & & & & & & & & & & \ar[dl] \ar[dr] & & \ar[dl] \ar[dr] & & \ar[dl] \ar[dr] \\
1 & 0 \ar[r] & & 1 \ar[r] & & 2\ar[r]  & & 3\ar[r] \ar@{-}[d]  & & 4 & & 5 & & 6 & & 7\ar[r]  & & 8\ar[r]  \ar@{-}[d]  & & 9\ar[r]  & & 10\ar[r]  & & 11 \\
  & 0 \ar[r] & & 0 \ar[r] & & 0\ar[r]  & & 0\ar[r] \ar@{-}[d]  & & 1 & & 2 & & 4 & & 6\ar[r]  & & 12\ar[r]  \ar@{-}[d]  & & 12\ar[r]  & & 12\ar[r]  & & 12 \\
  &  & & & & & & & & & & & & & & & & & & & & & & 
}


\end{document}

